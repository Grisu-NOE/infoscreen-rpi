\section{Änderungen im Detail}
Im vorhergehenden Abschnitt wurde ein Script verwendet, um alle Einstellungen anzupassen.
In diesem Kapitel werden die vorgenommen Einstellungen erläutert (es muss davon nichts händisch durchgeführt werden, da die Änderungen bereits durch das Script vorgenommen wurden).\\
Hinweis: Die \refsec{zeitzone}, \refsec{sprache} und \refsec{tastatur} könnten auch im Menü zur Einrichtung des Betriebsystems (siehe \refsec{stepsinstall}) vorgenommen werden.
Sie wurden in das Script eingefügt, weil teilweise die Einrichtung dieser Punkte im Menü nicht funktionierte.

\subsection{Zeitzone einrichten}
\label{sec:zeitzone}
Die Zeitzone wird auf \begin{em}Europe/Vienna\end{em} gestellt.
Dazu wird die entsprechende Zeitzone in die Datei \lstinline|/etc/timezone| geschrieben und mit dem Befehl \lstinline|dpkg-reconfigure tzdata| angewendet.

\subsection{Systemzeit aktualisieren}
\label{sec:zeitsync}
%Der Server \lstinline|ts1.univie.ac.at| wird in der Datei \lstinline|/etc/ntp.conf| zur Zeitsynchronisierung eingetragen.
%(Die Default-Server hatten in seltenen Fällen Probleme mit der Zeitsynchronisierung.)
%Der Zeitsynchronisierung-Dienst wird gestoppt, danach mit dem Befehl \lstinline|ntpd -gqx| eine sofortige Synchronisierung durchgeführt und der Dienst wieder gestartet.
(Nicht mehr notwendig, da aktuell \lstinline|timedatectl| von \lstinline|systemd| in Verwendung ist.)

\subsection{Sprache einrichten}
\label{sec:sprache}
Die Zeitzone (bzw. \begin{em}locale\end{em}) wird auf \begin{em}de\_AT.UTF-8\end{em} (bedeutet Deutsch mit Zeichencodierung UTF-8) gestellt.
Dazu wird der entsprechende Eintrag in der Datei \lstinline|/etc/locale.gen| aktiviert und mit dem Befehl \lstinline|locale-gen| und \lstinline|update-locale| angewendet.

\subsection{Tastaturlayout einrichten}
\label{sec:tastatur}
Das Tastaturlayout wird auf deutsch gestellt (z.B. sind \lstinline|Z| und \lstinline|Y| im vergleich zum englischen Tastaturlayout vertauscht).
Dazu wird der entsprechende Eintrag in die Datei \lstinline|/etc/default/keyboard| geschrieben und mit dem Befehl \lstinline|dpkg-reconfigure keyboard-configuration| angewendet.

\subsection{Paketinstallation}
\label{sec:paketinstallation}
Folgende Software-Pakete werden (nach einem Update der Software-Paketliste) installiert:
\begin{itemize}
	\item {\lstinline|firefox-esr|\\Dies ist der verwendete Web-Browser um die Infoscreen-Webseite anzuzeigen (\begin{em}Firefox\end{em} wird in Debian auch teilweise \begin{em}Iceweasel\end{em} genannt, details dazu siehe \cite{iceweasel}).}
	\item {\lstinline|unclutter|\\Wird benötigt, um den Mauszeiger auszublenden.}
%	\item {\lstinline|x11-xserver-utils|\\Wird benötigt, um den Bildschirmschoner und den Bildschirm-Standby zu deaktivieren.}
	\item {\lstinline|libjsoncpp1|\\Wird für Firefox benötigt.}
	\item {\lstinline|xmlstarlet|\\Wird zur automatischen Installation von Firefox-Plugins benötigt.}
\end{itemize}

Diese Pakete werden nicht aus dem raspbian-Repository installiert, da in der aktuellen Firefox-Version die Einrichtung des Kiosk-Modus Probleme bereitete.
Die installierten Softwarepakete beinhalten den Firefox (bzw. Iceweasel) in einer älteren Version. 
Bei der Verwendung von älterer Software ist jedoch besonders darauf Bedacht zu nehmen, dass diese sicherheitskritische Schwachstellen enthalten kann.
Der Zugriff auf das RaspberryPi sollte daher (wenn möglich) im Netzwerk eingeschränkt werden und der Browser nicht zum Surfen auf anderen Webseiten verwendet werden.

\subsection{Firefox Plugin}
%Um den Firefox automatisch im Vollbild zu starten wird das Plugin \begin{em}R-kiosk\end{em} \cite{rkiosk} installiert.
Um den Firefox automatisch im Vollbild zu starten wird das Plugin \begin{em}Full Screen Multiple Monitor\end{em} \cite{mkiosk} installiert.

\subsection{Autostart}
\label{sec:autostartconf}
Die Datei \lstinline|/home/pi/.config/lxsession/LXDE-pi/autostart| wurde wie folgt angepasst:\\
Original-Inhalt:
\begin{lstlisting}
@lxpanel --profile LXDE-pi
@pcmanfm --desktop --profile LXDE-pi
@xscreensaver -no-splash
\end{lstlisting}
Neuer Inhalt:
\begin{lstlisting}
@lxpanel --profile LXDE-pi
@pcmanfm --desktop --profile LXDE-pi
#@xscreensaver -no-splash

@xset s off
@xset -dpms
@xset s noblank

@firefox

@unclutter -idle 0.1 -root
\end{lstlisting}
\begin{itemize}
%	\item {Die Zeilen 1 bis 3 wurden mit dem \lstinline|#|-Zeichen auskommentiert. (Der Start des normalen Desktops und des Bildschirmschoners wird damit nicht mehr ausgeführt.)}
	\item {Die Zeile 3 wurde mit dem \lstinline|#|-Zeichen auskommentiert. (Der Start des Bildschirmschoners wird damit nicht mehr ausgeführt.)}
	\item {Die Zeilen 5 bis 7 sorgen dafür, dass der Bildschirm nie abgeschaltet und kein Engergiesparmodus aktiviert wird.}
	\item {Zeile 9 startet den Browser.}
	\item {Zeile 11 sorgt dafür, dass der Mauszeiger versteckt wird.}
\end{itemize}

\subsection{Browser-Konfiguration}
\label{sec:browserconf}
Zuserst werden alle Firefox-Profile gelöscht und anschließend wird mit dem Befehl \lstinline|firefox -CreateProfile Infoscreen| ein leeres Firefox-Profil angelegt.
In der Datei \lstinline|/home/pi/.mozilla/firefox/*Infoscreen/prefs.js| werden folgende Optionen eingefügt:\\
\begin{itemize}
	\item {\lstinline|user_pref("browser.startup.homepage", "https://infoscreen.florian10.info/");|\\
		definiert die Startseite, welche geladen werden soll.}
	\item {\lstinline|user_pref("browser.cache.disk.capacity", 0);|\\
		deaktiviert den Cache des Browsers.}
	\item {\lstinline|user_pref("app.update.enabled", false);|\\
		deaktiviert Update-Meldungen.}
	\item {\lstinline|user_pref("general.useragent.override", "Rpi-Infoscreen2.3");|\\
		setzt einen individuellen User-Agent.}
	\item {\lstinline|user_pref("browser.sessionstore.enabled", true);| und \\
		\lstinline|user_pref("browser.sessionstore.resume_from_crash", false);| und \\
		\lstinline|user_pref("browser.sessionstore.browser.sessionstore.max_resumed_crashes", 0);|\\
		verhindern, dass nach einem ungeplanten Neustart von Firefox ein Hinweis zur Tab-Wiederherstellung angezeigt wird.}
\end{itemize}

\subsection{Desktop-Verknüpfungen}
\label{sec:desktopconf}
Am Desktop werden zwei Verknüpfungen zu Firefox angelegt.
Die Datei \lstinline|/home/pi/Desktop/firefox.desktop| sorgt für eine Verknüfung zum normalen Start von Firefox:\\
\begin{lstlisting}
[Desktop Entry]
Encoding=UTF-8
Name=Firefox
Exec=firefox-esr
Terminal=false
Type=Application
Icon=firefox-esr
StartupNotify=true
\end{lstlisting}

Die Datei \lstinline|/home/pi/Desktop/safemode.desktop| sorgt für eine Verknüfung zum Start von Firefox im Safe-Mode (dieser unterbindet den Start von Plugins und damit dass Firefox im Vollbild-Modus gestartet wird):\\
\begin{lstlisting}
[Desktop Entry]
Encoding=UTF-8
Name=Firefox (safe-mode)
Exec=firefox-esr -safe-mode
Terminal=false
Type=Application
Icon=firefox-esr
StartupNotify=true
\end{lstlisting}

\subsection{automatischer Neustart}
\label{sec:rebootconf}
Die Datei \lstinline|/etc/crontab| wurde wie folgt angepasst:\\
Inhalt der Datei:
\begin{lstlisting}
SHELL=/bin/sh
PATH=/usr/local/sbin:/usr/local/bin:/sbin:/bin:/usr/sbin:/usr/bin

# m h dom mon dow user	command
17 *	* * *	root    cd / && run-parts --report /etc/cron.hourly
25 6	* * *	root	test -x /usr/sbin/anacron || ( cd / && run-parts --report /etc/cron.daily )
47 6	* * 7	root	test -x /usr/sbin/anacron || ( cd / && run-parts --report /etc/cron.weekly )
52 6	1 * *	root	test -x /usr/sbin/anacron || ( cd / && run-parts --report /etc/cron.monthly )

7  4	* * *	root	echo "`date` - periodic reboot" >> /root/reboot.log; reboot 
*/5  *	* * *	root	pidof firefox-esr > /dev/null && echo "`date` - firefox-esr running" > /root/running.log || (echo "`date` - firefox-esr not running" >> /root/reboot.log; reboot )
0  5	5 1 *	root	echo "`date` - cleaned file" > /root/reboot.log;

#
\end{lstlisting}
\begin{itemize}
	\item {Zeile 10 sorgt dafür, dass das RaspberryPi jeden Tag um 04:07 Uhr neu gestartet wird (das wird in der Datei \lstinline|/root/reboot.log| protokolliert).}
	\item {Zeile 11 sorgt dafür, dass das RaspberryPi neu gestartet wird, falls der Firefox aus unerwarteten Gründen beendet wird. Dies wird alle 5 Minuten geprüft.
		Es erfolgt ebenfalle eine Protokollierung der Ereignisse in der Datei \lstinline|/root/reboot.log|.}
	\item {Zeile 12 beinhaltet den Befehl die Protokoll-Datei \lstinline|/root/reboot.log| jährlich zu leeren.}
\end{itemize}

\subsection{Bildschirm-Standby deaktivieren}
Die Datei \lstinline|/etc/lightdm/lightdm.conf| wurde wie folgt angepasst:\\
Die originale Zeile:
\begin{lstlisting}
#xserver-command=X
\end{lstlisting}
Wird ersetzt durch die neue Zeile:
\begin{lstlisting}
xserver-command=X -s 0 -dpms
\end{lstlisting}
Diese Änderung deaktiviert den Bildschirm-Standby.\\
(Hinweis: Möglicherweise ist diese Änderung nicht notwendig. Bei einem Test funktionierte es auch mit der originalen Datei.)

\section{Anmerkungen / FAQ}
\label{sec:faq}
In diesem sollen häufig gestellte Fragen beantwortet werden bzw. einige Tipps zusammengefasst werden.

\subsection{GUI verwenden}
Beim Start wird auch automatisch Firefox im Vollbild-Modus gestartet. Falls man auf der grafischen Oberfläche arbeiten möchte, muss dieser mit \lstinline|ALT + F4| zunächst beendet werden.
Da aber das Raspberry nach spätestens 5 Minuten neu starten würde (siehe auch \refsec{rebootconf}) sollte der Firefox im Safe-Mode (siehe auch \refsec{desktopconf}) wieder gestartet werden (Desktop-Verknüpfung \lstinline|Iceweasel (safe-mode)| dann klick auf \lstinline|Start in Safe Mode|).
Damit wird Firefox nicht im Vollbild gestartet und kann minimiert werden.

\subsection{Internetverbindung nach Stromausfall}
\label{sec:stromausfall}
Nach einem Stromausfall hatte eine Feuerwehr das Problem, dass das Modem einige Zeit benötigt, bis wieder eine Internetverbindung hergestellt ist.
Wenn das Raspberry schneller startet, wird im Browser angezeigt, dass keine Internetverbindung gefunden wurde, die Seite aber nicht mehr neu geladen.

Die folgende Änderung in der Datei \lstinline|/etc/rc.local| kann bei so einem Problem Abhilfe schaffen:\\
Original-Inhalt (Ausschnitt, Auslassungszeichen \lstinline|[...]|):
\begin{lstlisting}
#!/bin/sh -e
# [...]
# By default this script does nothing.

# Print the IP address
_IP=$(hostname -I) || true
if [ "$_IP" ]; then
  printf "My IP address is %s\n" "$_IP"
fi

exit 0
\end{lstlisting}
Folgender Inhalt (Zeile 11) sollte hinzugefügt werden:
\begin{lstlisting}
#!/bin/sh -e
# [...]
# By default this script does nothing.

# Print the IP address
_IP=$(hostname -I) || true
if [ "$_IP" ]; then
  printf "My IP address is %s\n" "$_IP"
fi

ping 8.8.8.8 -c 3 && (echo "internet ok" >> /root/reboot.log;) || (echo "NO internet" >> /root/reboot.log; sleep 180 && reboot; )

exit 0
\end{lstlisting}
Diese Datei wird beim Start des Raspberry ausgeführt. Der Eintrag in Zeile 11 prüft beim Start die Internetverbindung (Ping eines Google-Servers). Falls dies fehlschlägt, wird 3 Minuten (180 Sekunden) gewartet und das Raspberry neu gestartet. (Wie die Datei geändert werden kann siehe \refsec{dateiaendern})

\subsection{Dateien als root ändern}
\label{sec:dateiaendern}
Am einfachsten kann eine Datei als root (vergleichbar mit Administrator-Benutzer) zu ändern ist per SSH.
\begin{itemize}
	\item {Informationen zur Verbindung siehe \refsec{stepssetup} bis zum Punkt \begin{em}Bei login as: ...\end{em}.}
	\item {Zum Start eines einfachen Editors folgenden Befehl eingeben:\\ \lstinline|sudo nano /etc/rc.local|\\
		Dieser Befehl hat folgende Bedeutung:\\
		\lstinline|sudo| - damit die Aktion als root ausgeführt wird\\
		\lstinline|nano| - Start eines einfachen Editors\\
		\lstinline|/etc/rc.local| - welche Datei im Editor geöffnet werden soll
		}
	\item {Mit \lstinline|STRG+X| wird der Editor beendet und \lstinline|Y| bestätigt ggf. das Speichern der Änderungen.
		(Wenn PuTTY verwendet wird, kann der Inhalt der Windows-Zwischenablage mit Klick der rechten Maustaste eingefügt werden - nicht mit \lstinline|STRG+Y|!)
		}
	\item {Weitere Informationen zur Bedienung des Editors nano:\\ \url{http://wiki.ubuntuusers.de/Nano}
		}
\end{itemize}

\subsection{Über dieses Dokument}
\label{sec:github}
Der Quellcode zu diesem Dokument ist auf GitHub verfügbar:\\
\url{https://github.com/Grisu-NOE/infoscreen-rpi}

\section{Änderungsprotokoll}
In diesem Abschnitt werden die Änderungen bei neuen Versionen aufgelistet, damit ggf. ein schneller Überblick über Neuerungen und Aktualisierungen möglich ist.
\begin{itemize}
	\item {Version vom 17. August 2014\\ Dokument erstellt.}
	\item {Version vom 26. September 2014\\ Dokument aktualisiert für eine neue Version des raspbian Images (\lstinline|2014-09-09-wheezy-raspbian.zip|). Funktionsumfang unverändert.\\
		Danke für die Hinweise von \textit{fuchsi} und \textit{schmiddy} im wax.at-Forum \cite{wax1}.}
	\item {Version vom 28. Februar 2015\\ Dokument aktualisiert für das neue das \begin{em}RaspberryPi 2 Model B\end{em}. Version des raspbian Images (\lstinline|2015-02-16-raspbian-wheezy.zip|) aktualisiert. Verwendung des Browsers \textit{Firefox} statt bisher \textit{midori}. Sonstiger Funktionsumfang unverändert.\\
		Danke für die Hinweise und Anmerkungen von einigen Kameraden, die mich per Email erreicht haben.
		}
	\item {Version vom 09. März 2015\\ Folgende Änderungen: 
		Firefox Tab-Wiederherstellung abgeschaltet (siehe \refsec{browserconf}). 
		Desktop-Verknüpfungen zum manuellen Start von Firefox hinzugefügt (siehe \refsec{desktopconf}).
		In der letzten Version war die Taskleiste ausgeblendet. Dieser Fehler wurde behoben (siehe \refsec{autostartconf}).
		Neuer Abschnitt \textit{Anmerkungen / FAQ} (siehe \refsec{faq}).\\
		Hinweis: Falls eine Aktualisierung von der Version vom 28. Februar 2015 vorgenommen werden soll, 
		reicht es die Schritte ab \refsec{stepssetup} \textit{Einrichtung des Infoscreens} auszuführen.
		(Das \refsec{stepsinstall} \textit{Installation des Betriebsystems} muss nicht erneut durchgeführt werden.)
		}
	\item {Version vom 08. Juni 2015\\ Keine Änderungen im Script.
		Hinweise in FAQ-Bereich eingefügt: Internetverbindung nach Stromausfall (siehe \refsec{stromausfall}) und Dateien als root ändern (siehe \refsec{dateiaendern}).
 		}
	\item {Version vom 04. Jänner 2016\\ 
		Modell \textit{Zero} in Tabelle und Hinweis auf Hardwareübersicht in Wikipedia eingefügt (siehe \refsec{rpi}).
		Die Version des Installers hat sich (\lstinline|2015-05-05-raspbian-wheezy.zip| bzw. \lstinline|2015-11-21-raspbian-jessie.zip|) geändert. 
		Anpassungen des Scripts für die neue Version (siehe auch \refsec{zeitzone}, \refsec{zeitsync}, \refsec{sprache} und \refsec{tastatur}).
		Funktionsumfang unverändert.
 		}
	\item {Version vom 09. Mai 2016\\ 
		Modell \textit{RaspberryPi 3 Model B} in Tabelle eingefügt (siehe \refsec{rpi}).
		Die Version des Installers hat sich (\lstinline|2016-03-18-raspbian-jessie.zip|) geändert. 
		Anpassungen des Scripts für die neue Version.
		Funktionsumfang unverändert.
		Hinweis zu WLAN (siehe \refsec{rpiwlan}) und auf GitHub (siehe \refsec{github}) eingefügt.
 		}
	\item {Version vom 07. August 2016\\ 
		Die Version des Installers hat sich (\lstinline|2016-05-27-raspbian-jessie.zip|) geändert. 
		Anpassungen des Scripts für die neue Version.
		Funktionsumfang unverändert.
		Hinweise der Paketinstallation (siehe \refsec{paketinstallation}) ergänzt.
 		}
	\item {Version vom 03. Februrar 2017\\ Keine Änderungen im Script.
		URL zum Download des Images in \refsec{stepsinstall} angepasst.
 		}
	\item {Version vom 20. März 2017\\ 
		Die Version des Installers hat sich (\lstinline|2017-03-02-raspbian-jessie.zip|) geändert. 
		Anpassungen des Scripts für die neue Version.
		Funktionsumfang unverändert.
 		}
	\item {Version vom 26. Juli 2019\\ 
		Modell \textit{RaspberryPi 4 Model B} in Tabelle eingefügt (siehe \refsec{rpi}).
		Die Version des Installers hat sich (\lstinline|2019-07-10-raspbian-buster.zip|) geändert. 
		Anpassungen des Scripts für die neue Version (neue Firefox-Version, alternatives Kiosk-Mode-Plugin).
		Funktionsumfang unverändert.
 		}
\end{itemize}

