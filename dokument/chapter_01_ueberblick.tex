\section{Vorwort}
Infoscreens werden von immer mehr Feuerwehren eingesetzt, um Einsatz-Informationen im Feuerwehrhaus anzuzeigen.
Dank einem Service der FF Krems \cite{wastl} und immer günstigerer Hardware ist die in Betriebnahme eines Infoscreens keine unlösbare Aufgabe mehr.
Dieses Dokument soll Tipps für die Einrichtung eines kostengünstigen Infoscreens geben, da es an mich als BSB EDV bereits einige Anfragen dazu gab.

Die folgenden Informationen wurden nach bestem Wissen und Gewissen zusammengestellt, erheben allerdings keinen Anspruch auf Richtigkeit und/oder Vollständigkeit.
Auch sind die Abläufe nicht als strikte Anweisungen zu verstehen, sondern sollen Hilfestellungen geben, 
da jeder Infoscreen natürlich den individuellen Anforderungen und ortlichen Gegebenheiten der jeweiligen Feuerwehr angepasst ist.

Für Anmerkungen, Fragen und Hinweise bitte um ein Email an: \verb|infoscreen@122.gv.at|

Ich ersuche ebenso die Benutzer dieser Anleitung um eine kurze Email, damit ich euch bei Aktualisierungen und Neuerungen direkt informieren kann.

\section{Wastl Infoscreen}
\label{sec:wastl}
Die Infoscreen-Anwendung ist als Web-Applikation verfügbar und wird von der FF Krems für Feuerwehren zur Verfügung gestellt \cite{wastl}.\\
Weitere Informationen siehe:\\
\url{http://www.feuerwehr-krems.at/ShowArtikel.asp?Artikel=8930}\\
Dort ist auch der notwendige Ablauf zur Freischaltung der eigenen Feuerwehr beschrieben.

An dieser Stelle möchte ich meinen Dank an die Entwickler der FF Krems richten, die dieses tolle Service möglich machen!


